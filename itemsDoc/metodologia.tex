%SECCIÓN 3. METODOLOGIA
\section{METODOLOGIA}

 Para el desarrollo de este documento; se utilizará algunas tareas de la metogología de análisis de BigData[1].
 
  \subsection{Reconocimiento de la información}
 
  \textbf{Identificar el dominio}: Se explorará el DataSet obtenido de Pew Research Center’s Hispanic Trends Project[2], en el cual se encuentran 12544 observaciones y diferentes variables de información, entre ellas la cantidad de ciudadanos hispanos, no hispanos y total de población que se ha encontrado en algunas ciudades de los Estados Unidos y como ha sido el comportamiento de los datos en los diferentes años de la muestra (1990, 2000, 2010, 2011).\\
   
  \textbf{Variables del DataSet:} Las variables que se identificaron en el conjunto de datos son las siguientes:    	 
   	\begin{itemize}
   	\item \textbf{COUNTY}: Ciudad de un estado.%\\ 
    \item \textbf{STATE}: Estado de EEUU.%\\
   	\item \textbf{TP}: Total de población.%\\
   	\item \textbf{TPNH}: Total de población no Hispana.%\\
   	\item \textbf{TPH}: Total de población Hispana.%\\
   	\item \textbf{PPH}: Porcentaje de población Hispana.%\\
   	\item \textbf{AP}: Año de la población.%\\
   \end{itemize}
   
   \textbf{Identificar un problema}: El crecimiento poblacional hispano que ha tenido EEUU en los últimos años[2] es muy considerable; y debido al gran impacto socio-económico que esto puede acarrear en un futuro, se hace necesario poder estimar el crecimiento poblacional hispano en las diferentes ciudades principales de los EEUU. Con el desarrollo de esta investigación se propondrá un modelo predictivo que ayudará a solventar esta problematica.\\ 
    
   \textbf{Características de los objetivos SMART}: Los objetivos del proyecto de investigación deben ser orientados con las caracteristicas SMART[3], lo que significa que estos objetivos han de contemplar las siguientes cualidades:%\\
   \begin{itemize}
     \item Specific  (Específico): Dirigirse a un área específica de mejora.%\\
     \item Measurable (Medible): Cuantificar o al menos sugerir un indicador de progreso.%\\
     \item Attainable (Alcanzable): Identificar que tipo de habilidades, actitudes u otro tipo de recursos necesitamos para cumplirlas.%\\
     \item Realistic (Realista): Los resultados esperados son acordes con los rescursos disponibles.%\\
     \item Time-bound (Oportuno): Especificar un marco de tiempo para lograr el resultado.%\\               
	\end{itemize}

  \subsection{Tipos de preguntas de una investigación}
  Las preguntas de investigación[4] que se desarrollarán en el proyecto están enmarcadas en los siguientes ámbitos:
  \begin{itemize}
   \item Descriptivas: Una pregunta descriptiva es la que busca resumir una característica de un conjunto de datos.
   \item Exploratorias: Las preguntas de caracter exploratorio consisten en la busqueda de patrones o relaciones que soporten una pregunta de investigación.
   \item Inferenciales: Una pregunta inferencial consiste en el planteamiento de una hipotesis que podria ser resuelta con el analisis respectivo de la informacion.
   \item Predictivas: Las preguntas de caracter predictivo permiten analizar el comportamiento de la informacion a traves del tiempo, con el objetivo de descubrir, proyectar, o realizar hipotesis sobre estados futuros.
  \end{itemize}
  
  
  \subsection{Marco teórico del análisis exploratorio}
  
  El análisis exploratorio de los datos son básicamente aquellas funciones estadisticas que permiten visualizar el comportamiento de las observaciones en el DataSet en un proceso de investigación. Las funciones a utilizar son las siguientes: 
  \begin{itemize}
  	%alzate
  	\item Experimento Aleatorio[5]; Es un proceso de observación mediante el cual se selecciona un elemento de un conjunto de posibles resultados. Un experimento aleatorio es aquel en el que él resultado no se puede predecir con anterioridad a la realización misma del experimento. 
  	
  	%alzate
   \item Frecuencia relativa[5]; Sea $A$ un subconjunto del conjunto de posibles resultados de un experimento aleatorio "llamado $\Omega$". Si repetimos $N$ veces el experimento y observamos que en $N_{A}$ de esas repeticiones se obtuvo un elemento de $A$, decimos que $f_{N}(A)=\frac{N_{A}}{N}$ es la frecuencia relativa del subconjunto $A$ en esas $N$  repeticiones del experimento.

	%canavos
    \item Medidas de tendencia central[6]; 
	   \begin{itemize}
		 	\item Media: la media de las observaciones de un experimento aleatorio $x_{1},x_{2},.....x_{n}$ es el promedio aritm\'etrico de \'estas y se denota por;
		 	$$\overline{x}=\sum_{i=1}^{n} \frac{X_{i}}{n}$$ 
		 	\item Moda: la moda de un conjunto de observaciones de un experimento aleatorio es el valor de la observaci\'on que ocurre con mayor frecuencia en el conjunto.
		 	%14
		 	\item Mediana: la mediana repreesenta el valor de la variable de posición central en un conjunto de datos ordenados de un experimento aleatorio.
		 \end{itemize}
 
	 \item Varianza[6]; La Varianza de las observaciones $x_{1},x_{2},...,x_{n}$ es en esencia, el promedio del cuadrado de las distancias entre cada observaci\'on y la media del conjunto de observaciones. Se denota por:
	 $$s^{2}=\sum_{i=1}^{n} \frac{ \left( x_{i}-\overline{x}\right)^{2}}{\left(n-1 \right) } $$ 
 
	  \item Desviaci\'on est\'andar[6]; La desviaci\'on est\'andar es la raiz cuadrada de la varianza y se denota por:
	 $$s=\sqrt{\sum_{i=1}^{n} \frac{ \left( x_{i}-\overline{x}\right)^{2}}{\left(n-1 \right) } }$$ 
	 
	 \item Cuartiles[6]; Los cuartiles son dada una serie de valores  $x_{1},x_{2},...,x_{n}$ ordenados en forma creciente, podemos pensar que su cálculo podría efectuarse:
		 \begin{itemize}
		 	\item Primer cuartil (Q1) como la mediana de la primera mitad de valores.
		 	\item Segundo cuartil (Q2) como la propia mediana de la serie de valores.
		 	\item Tercer cuartil (Q3) como la mediana de la segunda mitad de valores.
		\end{itemize}
	\end{itemize}  
