%SECCIÓN 1. INTRODUCCIÓN 
\section{Introducción}
	Los modelos predictivos hoy en día a nivel mundial deben ser parte fundamental en el desarrollo y crecimiento de las organizaciones; sin tener en cuenta el tipo de actividad que realizan, ya que a través de estos modelos se puenden extraer patrones de los datos históricos y transaccionales con el objetivo de identificar riesgos y oportunidades de negocio. El análisis predictivo agrupa una variedad de técnicas estadísticas de modelización, aprendizaje automático y minería de datos; la cuales  a través de los datos históricos y actuales permiten realizar predicciones acerca del futuro o acontecimientos no conocidos. Teniendo en cuenta lo anterior, y tomando como base para el desarrollo del estudio el DataSet extraído de Pew Research Center y la aplicación de la metodología de análisis de BigData, se analizará el comportamiento poblacional hispano en los diferentes años (1990, 2000, 2010 y 2011), con el objetivo de proponer un modelo predictivo que permita estimar el crecimiento poblacional hispano que tendrán diferentes ciudades de EEUU en el año 2020.